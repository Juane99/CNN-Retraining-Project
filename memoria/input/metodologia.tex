\section{Metodología utilizada}


\subsection{Base de datos utilizada}

\vspace{5 mm}

A continuación hablaremos sobre la base de datos que hemos utilizado, los inconvenientes que hemos encontrado al usarla y como los hemos solucionado.

La base de datos que hemos usado se llama \textbf{Vietnamese Foods} y la subío un estudiante de la VNU HCMC a la web \textbf{kaggle}. Este dataset consta de 24474 imágenes divididas en 29 clases. Las imágenes no fueron previamente divididas en un set de entrenamiento y un set de test. Tampoco disponíamos de un archivo de texto con la ruta de todas las imágenes del dataset por lo que tuvimos que construirlo a mano usando la orden \textbf{find} que nos proporciona el terminal de Linux.

Una vez obtenido dicho archivo de texto con las imágenes nos fue sencillo construir los correspondientes sets de entrenamiento y test. Tuvimos que cargar todas las imágenes con sus respectivas clases en dos ndarrays, y usarlos en la función \textbf{train\_test\_split} de sklearn con un 0.2 de tamaño de test. Como las clases que usamos eran de tipo string, tuvimos que hacer uso también de la función \textbf{to\_categorical} para convertirlas en valores enteros.

Una vez intentamos cargar las imágenes en memoria nos dimos cuenta que no era posible debido al costo computacional que requería almacenar tantas imágenes en la RAM. Por este motivo redujimos el número de clases de forma aleatoria hasta llegar a \textbf{9}, que era el máximo número de clases que podíamos cargar en Google Colab sin sobrepasar sus recursos computacionales. Cabe destacar que todas las clases del dataset están equilibradas. A continuación mostraré ejemplos de las diferentes imágenes que se pueden encontrar en el dataset usado:

\vspace{5 mm}

\begin{figure}[H]
  \centering
  \includegraphics[width=0.5\linewidth]{Imagenes/comida-vietnamita.png}
  \caption{Imágenes del dataset Vietnamese Foods}
  \label{fig:sub-first}
\end{figure}

\newpage

\subsection{Preprocesado de los datos}


\subsection{Redes neuronales convolucionales entrenadas con ImageNet utilizadas}

\vspace{5 mm}



\subsection{Mejoras sobre las redes utilizadas}
